% Farbe

\def \customColor {PineGreen}

% Bullet-Symbol für Aufzählungen
\renewcommand\textbullet{\ensuremath{\bullet}}

% Eingekreiste Nummern für Aufzählungen
\newcommand*\circled[1]{\tikz[baseline=(char.base)]{
        \node[shape=circle,draw,inner sep=1.2pt] (char) {#1};}}

% Schriftart
\renewcommand{\familydefault}{\sfdefault}

%Tabelle Horiz. Grösse
\renewcommand{\arraystretch}{1.5}

%Einheiten
\newcommand{\Einheit}[1]{
$\bigl[ #1 \bigr]$
}

\newcommand{\EinheitBruch}[2]{
$\bigl[ \frac{#1}{#2} \bigr]$
}

\newcommand{\Umbruch}{\vfill\null\columnbreak}

% Titel-Block	
\newcommand{\Header}[3]{
	\begin{tcolorbox}  [arc = 0mm, 
	                    colback = \customColor!50!black, 
	                    colframe = \customColor!50!black, 
	                    valign = center, 
	                    fontupper=\color{white}]
		\large \center \textbf{#1} \par
		\huge \textbf{#2} \par 
		\vskip 5pt 
		\large #3 \par
		\vskip 3pt 
		\small Version: \today
	\end{tcolorbox}
	}

%Teil-Block
\newcommand{\Abschnitt}[1]{
	\begin{tcolorbox} [arc = 0mm,
						colback = \customColor!50!black,
						colframe = \customColor,
						valign = center,
						before skip = 3mm,
						leftright skip = -0.5mm,
						after skip = 1 mm,
						bottomrule = 0 mm,
						toprule = 0 mm,
						leftrule = 0 mm,
						rightrule = 0 mm,
						fontupper=\color{white}
						]
		\centering\Large\textbf{#1}
	\end{tcolorbox}
}

% Überschrift
\renewcommand{\section}[1]{
	\begin{tcolorbox}[
			arc=0mm,
			colback=\customColor!50!black,
			colframe=white,
			bottomrule = 0 mm,
			toprule = 0 mm,
			leftrule = 0 mm,
			rightrule = 0 mm,
			valign=center,
			left=0.5mm,
			top= 0.7 mm,
			bottom= 0.7 mm,
			fontupper=\color{white},
			before skip = 1mm,
			leftright skip = -0.5mm,
			after skip = 1 mm]

		\textbf{#1}
	\end{tcolorbox}
	}

% Abschnitt	
\renewcommand{\subsection}[1]{
\begin{tcolorbox}[
            		arc=0mm,
            		colback=\customColor!50,
            		colframe=white,
            		bottomrule = 0 mm,
            		toprule = 0 mm,
            		leftrule = 0 mm,
            		rightrule = 0 mm,
            		valign=center,
            		left=0.5mm,
            		top=0.2mm,
            		bottom=0.2mm,
            		before skip = 1mm,
            		leftright skip = -0.5mm,
            		after skip = 1mm]
	\small \textbf{#1}
\end{tcolorbox}
}

\renewcommand{\subsubsection}[2]{
\begin{tcolorbox}[
            		arc=0mm,
            		colback=gray!50,
            		colframe=white,
            		bottomrule = 0 mm,
            		toprule = 0 mm,
            		leftrule = 0 mm,
            		rightrule = 0 mm,
            		valign=center,
            		left=0.5mm,
            		top=0.2mm,
            		bottom=0.2mm,
            		before skip = 1mm,
            		leftright skip = -0.5mm,
            		after skip = 1mm]
	\small #1 \hfill #2
\end{tcolorbox}
}

%MATH-BOX, optional commands in {}
\newtcbox{\mathbox}[1][]{%
    nobeforeafter, 
    tcbox raise base, 
    colframe=blue!30!black,
    colback=blue!20, 
    boxrule=1pt,
    arc=1mm,
    before skip = 0mm,
    right = 1mm,
    left=1mm,
	top=0.5mm,
	bottom=0.5mm,
    #1}
    
\newtcbox{\mathboxnoback}[1][]{%
    nobeforeafter, 
    tcbox raise base, 
    colframe=black,
    colback=white,
    boxrule=1pt,
    arc=1mm,
    before skip = 0mm,
    right = 1mm,
    left=1mm,
	top=0.5mm,
	bottom=0.5mm,
    #1}
    
\newcommand{\textbox}[1]{
\tcbox[
		arc=1mm,
		colback=blue!20,
		colframe=blue!30!black,
		arc=1mm,
		boxrule=1pt,
		right = 1mm,
        left=1mm,
	    top=0.5mm,
	    bottom=0.5mm,
		before skip = 1mm,
		after skip = 2 mm
		]
{    
	#1
}
}

\newcommand{\textboxmithead}[2]{
\begin{tcolorbox}[
		arc=1mm,
		colback=blue!20,
		colframe=blue!30!black,
		arc=1mm,
		boxrule=1pt,
		left=1mm,
		top= 1 mm,
		bottom= 1 mm,
		before skip = 2mm,
		leftright skip = 8mm,
		after skip = 2 mm
		]
    \centering{\textbf{#1}\par}
	#2
\end{tcolorbox}
}

\newtcolorbox{titlebox}[1][]{
        fonttitle=\bfseries, 
        colbacktitle=gray!80,
        enhanced,
        attach boxed title to top center={yshift=-2mm},
        colframe=black,
        colback=white,
        boxrule=1pt,
        arc=1mm,
        before skip = 2mm,
        right = 1mm,
        left=1mm,
    	top=0.5mm,
    	bottom=0.5mm,
        title=#1}